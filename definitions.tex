%--------------
% Metric Space
%--------------
\begin{defn}
A \underline{metric space} $(X,d)$ is a set $X$ and a function $d$(the distance function) which assigns a real number $d(x,y)$ to every pair $(x,y) \in X$, which satisfies the following properties :

\begin{enumerate}
\item $d(x,y) \ge 0$
\item $d(x,y) = 0 \Rightarrow x=y$.
\item $d(x,y) = d(y,x)$.
\item $d(x,y) + d(y,z) \ge d(x,z)$.  This last property is called the \textit{triangle inequality}.
\end{enumerate}
\end{defn}

%--------------
% Topological Transitivity
%--------------
\begin{defn}
A function $f$ is \underline{topologically transitive} iff for all nonempty open subsets $U, V$ of $X$, there exists $k \in \mathbb{N}$ such that $f^k(U) \cap V$ is nonempty.
\end{defn}

%--------------
% Dense set
%--------------
\begin{defn}
Let $X$ be a topological space.  A set $Q$ is \underline{dense} in $X$ if for any point $x \in X$ and for any $\epsilon > 0$, there exists a point in $q \in Q$ such that the distance between $x$ and $q$ is less than $\epsilon$.  In other words, a set $Q$ is dense in $X$ if every point in $X$ is either in $Q$ or is a limit point in $Q$.
\end{defn}

%--------------
% Periodic point
%--------------
\begin{defn}
A point x is said to be a \underline{periodic point} of a function $f$ if there exists an integer $n$ such that $f^n(x) = x$.  The least positive integer $n$ for which this is true is the period of $x$.
\end{defn}

%--------------
% Devaney's definition of chaos
%--------------
\begin{defn}
Let $(X,d)$ be a metric space.  A function $f:X \to X$ is said to be \underline{chaotic} on $X$ if it satisfies the following three conditions:
\begin{enumerate}
\item \textit{$f$ is topologically transitive}.  
\item \textit{The set of periodic points in $f$ is dense in $X$}.  That is, that every open set in $f$ contains a periodic point.
\item \textit{$f$ has sensitive dependence on initial conditions}.  That is, $\exists \delta > 0$ such that for any open set $U$ and for any $x \in U$, there exists a $y \in U$ such that $d(f^{[k]}(x), f^{[k]}(y)) > \delta$ for some k.  $\delta$ is called a \textit{sensitivity constant}.
\end{enumerate}

\end{defn}

%--------------
% Backward shift operator
%--------------
\begin{defn}
A \underline{backward shift operator} $B$ operates on an element of a sequence to produce the previous element.

e.g. if $X = \{x_1, x_2, \dots\}$, then
$B(X) = \{x_2, x_3, \dots\}$.
\end{defn}

%--------------
% Complex modulus
%--------------
\begin{defn}
Let $z \in \mathbb{C}$.  That is, let $z = x + yi$, where $x$ and $y$ are real numbers.  The \underline{absolute value} or \underline{modulus} of $z$, denoted $|z|$ is given by \[|z| = \sqrt{x^2 + y^2}.\]
\end{defn}

%--------------
% Open unit disc of C
%--------------
\begin{defn}
The \underline{open unit disc of $\mathbb{C}$}, denoted $\mathbb{D}$, is the region in the complex plane defined by \[\mathbb{D} = \{z \in \mathbb{C} : |z| < 1\}.\]
\end{defn}

%--------------
% Linear transformation/operator
%--------------
\begin{defn}
A mapping $T$ from a vector space $V_1$ to a vector space $V_2$, i.e. $T:V_1 \rightarrow V_2$, is a \underline{linear transformation} iff
\[T(c\vec{u} + c\vec{v}) = cT(\vec{u}) + cT(\vec{v}),\] for all $\vec{u}, \vec{v} \in V_1$, and all $c \in \mathbb{R}$.  The transformation is referred to as an \underline{operator} if the mapping is from a vector space to itself.
\end{defn}

%--------------
% Holomorphic/analytic function
%--------------
\begin{defn}
Let $U \subset \mathbb{C}$ be open and let $f: U \rightarrow \mathbb{C}$.  If $f$ is complex differentiable at every point in $U$, $f$ is said to be \underline{holomorphic} or on $U$.
\end{defn}

%--------------
% Pole of complex function
%--------------
\begin{defn}
A function $f$ has a \underline{pole} of order $n$ at $z_0$ if $n$ is the smallest positive integer for which $(z-z_0)^nf(z)$ is holomorphic at $z_0$.  A function $f$ has a pole at infinity if $\lim_{z\rightarrow \infty} f(z) = \infty$.
\end{defn}

%--------------
% Linear Fractional Transformation
%--------------
\begin{defn}
A \underline{Linear Fraction Transformation} is a function of the form
\[f(z) = \frac{az + b}{cz + d}\]
where $a,b,c,d \in \mathbb{C}, ad \neq cb$.
\end{defn}

%--------------
% Boundary of a circle
%--------------