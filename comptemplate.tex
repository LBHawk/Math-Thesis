\documentclass[11pt]{article}
\usepackage{amssymb,amsmath,amsthm}
\usepackage{setspace}
\usepackage{graphicx}
\usepackage{url}
\usepackage{nth}
%\setlength{\parindent}{0.0in}

\theoremstyle{plain}
\newtheorem{theorem}{Theorem}[section]
\newtheorem{prop}[theorem]{Proposition}
\newtheorem{lemma}[theorem]{Lemma}
\newtheorem{corollary}[theorem]{Corollary}
\newtheorem{conjecture}[theorem]{Conjecture}

\theoremstyle{definition}
\newtheorem{defn}[theorem]{Definition}
\newtheorem{exer}[theorem]{Exercise}
\newtheorem{remark}[theorem]{Remark}
\newtheorem{example}[theorem]{Example}

\newcommand*\conj[1]{\bar{#1}}
\newcommand{\overbar}[1]{\mkern 1.5mu\overline{\mkern-1.5mu#1\mkern-1.5mu}\mkern 1.5mu}
\newcommand{\etal}{\mbox{\emph{et al.\ }}}
\newcommand{\ie}{\mbox{\emph{i. e.\ }}}
\newcommand{\etals}{\mbox{\emph{et al.}'s\ }}
\newcommand{\authorsetals}{Jim{\`e}nez-Mungu{\`i}a \etals}
\newcommand{\authors}{Jim{\`e}nez-Mungu{\`i}a\ }

\setlength{\emergencystretch}{4pt}


\title{\bf{A Classification of Chaos for M{\"o}bius} Transformations of Shifts}
\begin{document}
\author {A Senior Comprehensive Project \\
by\\
Lucas B. Hawk \\
Allegheny College\\
Meadville, PA}

\maketitle
\thispagestyle{empty}
\begin{center}
Submitted to the Department of Mathematics in partial fulfillment of the requirements for the degree of Bachelor of Science.
\end{center}

\vspace{0.5 in}
\begin{center}
Project Advisor:  Dr. Brent Carswell \\
Second Reader: Dr. Rachel Weir
\end{center}

\vspace{0.5 in}
\begin{center}
I hereby recognize and pledge to fulfill my responsibilities, as defined in the Honor Code, and to maintain the integrity of both myself and the College community as a whole.
\end{center}

\vspace{0.25 in}
\begin{center}
Pledge:

\vspace{0.5 in}
-----------------------------------------------------------------

\vspace{0.00 in}
name
\end{center}

\doublespacing
\newpage

\thispagestyle{empty}

\begin{abstract}
We overview and expand on the results of a paper by  \authors \etal titled \textit{Chaos for Linear Fractional Transformations of Shifts}.  \authors \etal provide a characterization of chaos for $\varphi(B)$ on Banach sequence spaces, where $\varphi$ is a Linear Fractional Transformation and $B$ is the backward shift operator.  Before exploring this characterization, we first review some basic concepts from topology, complex analysis, and linear algebra.  Furthermore, we explore more advanced topics such as conformal mappings in $\mathbb{C}$, $l^p$ and Banach spaces, and chaos.
\end{abstract}
\newpage
\thispagestyle{empty}
\tableofcontents

\newpage
\doublespacing
\setcounter{page}{1}
\section{Introduction}
%\hspace{0.5 in}

This purpose of this comp is to show that I've learned something really wonderful.

As a bonus, when I'm done and have passed the oral exam, I will finally graduate!

\newpage
\section{Preliminaries}
The purpose of this section of the paper is to review and expand on some of the more basic mathematical topics which are important in \authorsetals paper.  We review topics in metric spaces, complex analysis, and a few things from linear algebra.  These topics are the basic building blocks we need to understand some of the more advanced concepts in the paper.

\subsection{Metric Spaces}
In this section we overview some elementary topological concepts about metric spaces.  Becoming comfortable with these concepts is an important step towards understanding the framework of Mungu{\`i}a \etals paper.  We specifically talk about metric spaces, norms, and the completion of metric spaces.  Finally, we touch on the concept of Banach spaces, which are particularly important in the paper.  Note that many definitions, theorems, etc. come from Gerald Edgar's \textit{Measure, Topology, and Fractal Geometry} and Kaplansky's \textit{Set Theory and Metric Spaces}.

\begin{defn}
A \textbf{metric space} is a set $S$ together with a function $d:S \times S \rightarrow [0, \infty)$ satisfying the following:
\begin{align*}
\textrm{(1)}\; d(x,y) &= 0 \Leftrightarrow x = y\\
\textrm{(2)}\; d(x,y) &\geq 0 \textrm{ for all } x,y \in X\\
\textrm{(3)}\; d(x,y) &= d(y,x)\\
\textrm{(4)}\; d(x,z) &\leq d(x,y) + d(y,z) \textrm{ (Triangle inequality)}
\end{align*}
The nonnegative real number $d(x,y)$ is called the \textit{distance} between $x$ and $y$, while the function $d$ itself is known as the \textit{metric} of the set $S$.  A metric space is written as $(S, d)$, but oftentimes the metric is implied and the space is simply referred to as $S$.  
\end{defn}

\begin{example}
The set of real numbers $\mathbb{R}$, with $d: \mathbb{R} \times \mathbb{R} \rightarrow \mathbb{R}$ defined by
\[d(x,y) = |x - y| \]
is a metric space.  This is the usual metric used with $\mathbb{R}$.
\end{example}

The complex plane $\mathbb{C}$ has a similar usual metric:

\begin{example}
The complex numbers $\mathbb{C}$ with $d: \mathbb{C} \times \mathbb{C} \rightarrow \mathbb{R}$ defined by
\[d(z,w) = |z - w| \]
where $|z|$ is the modulus of $z$ is a metric space.
\end{example}

Generally, algebraic operations are not defined on a metric space, just a distance function.  Meanwhile, a vector space (which is not necessarily a metric space) provides the operations of vector addition and scalar multiplication, but without a notion of distance.  We can combine a vector space with a \textit{norm}, though, to create a normed vector space --- note that all normed vector spaces are also metric spaces.

\begin{defn}
A \textbf{normed vector space} $(X, ||\cdot||)$ is a vector space $X$ with a function $||\cdot||:X \rightarrow \mathbb{R}$, called a \textit{norm} on $X$, such that for all $x,y \in X$ and $k \in \mathbb{R}$:
\begin{align*}
&\textrm{(1)}\; ||x|| \geq 0 \textrm{ and } ||x|| = 0 \textrm{ if and only if } x=0;\\
&\textrm{(2)}\; ||kx|| = |k|\,||x|| \textrm{ (scaling property)};\\
&\textrm{(3)}\; ||x + y|| \leq ||x|| + ||y||.
\end{align*}
\end{defn}

These properties are rather intuitive.  Property (1) says that a vector has nonnegative length, and the length of $x$ is 0 if and only if $x$ is the 0-vector; property (2) states multiplying a vector by a scalar $k$ multiplies its length by $k$; finally property (3) is the triangle inequality, which is analogous to property (4) of definition 3.1.

While a norm is defined rather similarly to a metric, the two are not the same.  However, we often define a metric on a normed metric space using the norm.

\begin{prop}
If $(X, ||\cdot||)$ is a normed vector space $X$, then $d: X \times X \rightarrow \mathbb{R}$, defined by $d(x,y) = ||x - y||$, is a metric on $X$.
\end{prop}

The necessary properties for $d$ to be a metric follow immediately from properties (1) and (3) of a norm.  If $X$ is a normed vector space, we always use the metric associated with its norm, unless specifically stated otherwise.

A metric defined on a norm has all the properties of a metric discussed earlier, as well as two more --- for all $x, y, z \in X$ and $k \in \mathbb{R}$
\[d(x+z, y+z) = d(x+y), \qquad d(kx, ky) = |k|d(x,y).\]

These properties are called \textit{translation invariance} and \textit{homogeneity}, respectively.  These properties are not included in a simple metric space because they do not even make sense in that framework --- recall that in a space which is only a metric space, we can not add points together or multiply them by scalars.

While there are a variety of norms which can be used on a vector space, the \textit{Euclidean norm} is the most common and perhaps the most intuitive.

\begin{example}
On $\mathbb{R}^n$, the length of a vector $x = (x_1, x_2, ..., x_n)$ is given by
\[||x := \sqrt{x_1^2+x_2^2+\cdots+x_n^2}.\]
This gives the distance from the origin to the point $x$ and is known as the \textit{Euclidean norm}.  It should be familiar as it is a result of the Pythagorean theorem.
\end{example}

\begin{example}
On $\mathbb{C}^n$, the most common norm is given by
\[||z|| := \sqrt{|z_1|^2 + |z_2|^2 + \cdots + |z_n|^2} = \sqrt{z_1\conj{z}_1 + z_2\conj{z}_2 + \cdots + z_n\conj{z}_n}\]
\end{example}

Note that while a metric is often derived from a norm, the existence of a metric does not imply a norm --- a metric does not even necessarily need to make geometric sense.  Take for example what is known as the \textit{discrete metric}:

\begin{example}
On any set $X$, the discrete metric is defined as
\[d(x,y)= 
\begin{cases} 
	0 & x=y, \\
	1 & x \neq y.\\
\end{cases}
\]
\end{example}
\newpage
\subsection{The Complex Plane}
\authorsetals work is done entirely in the complex plane.  In this section we review the complex numbers and some topological concepts of the complex plane.  Most definitions and theorems are borrowed from \textit{Fundamentals of Complex Analysis} by Saff and Snider \cite{Saff}.

If we picture the real numbers $\mathbb{R}$ as a one-dimensional number line, the set of complex numbers $\mathbb{C}$ can be thought of as a plane with $\mathbb{R}$ as its x-axis and the set of imaginary numbers as its y-axis.  Then, an element of $\mathbb{C}$  consists of both a real part and an imaginary part.

\begin{defn}
A \textbf{complex number} is an expression of the form $a + bi$ where $a, b \in \mathbb{R}$, and $i$ is defined as the imaginary number $\sqrt{-1}$.  Two complex numbers $a+bi$ and $c+di$ are said to be equal if and only if $a=c$ and $b=d$.
\end{defn}

So, for a complex number $z = a + bi$, $a$ is said to be the real part (denoted Re $z$) and $b$ is the imaginary part (denoted Im $z$).  With this notation we can write $z = \textrm{Re }z + i \textrm{ Im }z$.  Note that if $b=0$, then $z$ is a real number, while if $a=0$, then $z$ is a pure imaginary number.

Recall that the complex plane $\mathbb{C}$ can be combined with a distance function $d$:
\[d(z,w) = |z-w|\] where $z,w \in \mathbb{C}$ and $|z|$ is the modulus of $z$ defined as below.  Furthermore, this metric is induced by the usual norm of $\mathbb{C}$: $||z|| = \sqrt{x^2 + y^2}$ where $z = x+iy$ for $x,y \in \mathbb{R}$.  From this point on, this metric will be implied when we refer to $\mathbb{C}$ .

\begin{defn}
The \textbf{modulus} or (\textbf{absolute value}) of a number $z = a + bi$ is denoted $|z|$ and is given by
\[|z| := \sqrt{a^2 + b^2}\]
\end{defn}

Note that $|z|$ is always a nonnegative real number, and the only complex number whose modulus is zero is the number 0 itself.

The reflection of a point $z=a+bi$ across the real axis is the point $a-bi$.  The relationship between a number and its reflection plays a large role in complex analysis.

\begin{defn}
The \textbf{complex conjugate} of a number $z=a+bi$ is denoted $\conj{z}$ and is given by
\[\conj{z} := a-bi.\]
\end{defn}

The conjugate is important due to the numerous properties regarding the possible interactions between a complex number and its conjugate.

\begin{example}
First, it is clear that $z=\conj{z}$ if and only if $z$ is a real number.  Furthermore, the conjugate of the sum/difference of two complex numbers is equal to the sum/difference of their conjugates.  That is,
\[\overbar{z_1 + z_2} = \overbar{z_1} + \overbar{z_2}, \;\; \overbar{z_1 - z_2} = \overbar{z_1} - \overbar{z_2}.\]
Beyond these two properties are a number of other ones as well:
\begin{itemize}
\item $\overbar{(z_1z_2)} = \overbar{z_1}\,\overbar{z_2}$\\
Indeed, if $z_1 = a+bi$ and $z_2 = c+di$, then
\vspace{-1em}
\begin{align*}
\overbar{z_1z_2} &= \overbar{ac - bd + (ad + bc)i}\\
&= ac - bd - (ad + bc)i\\
&= ac - bd - adi - bci\\
&= (a - bi)(c-di)\\
&= \overbar{z_1}\,\overbar{z_2}.
\end{align*}
\item In addition, the following can be seen:
\[\overbar{\biggl(\frac{z_1}{z_2}\biggr)} = \frac{\overbar{z_1}}{\overbar{z_2}}, \; (z_2 \neq 0);\]
\[\textrm{Re } z = \frac{z+\overbar{z}}{2};\]
\[\textrm{Im } z = \frac{z-\overbar{z}}{2i}.\]
The last two properties demonstrate that the sum of a number and its conjugate is real, and the difference is imaginary, respectively.
\item The conjugate of a conjugate is, of course, the original number:
\[\overbar{(\overbar{z})} = z.\]
The final two properties are specifically used in \authorsetals paper:
\[|z| = |\overbar{z}|, \;\; z\overbar{z} = |z|^2.\]
The first of these is easy to see geometrically (Figure 1).

\begin{figure}[h]
\centering
\includegraphics[scale=0.4]{images/complexmod.png}
\caption{Complex number and its modulus\cite{modulusimage}}
\end{figure}

The second can be proved easily:
\[z\overbar{z} = (a+bi)(a-bi) = a^2 + b^2 = |z|^2.\]
\end{itemize}
\end{example}

For functions of a real variable, we typically work with functions defined on an \textit{interval}, but this concept does not work for $\mathbb{C}$.  We must instead define some topological concepts for $\mathbb{C}$.

\begin{defn}
The \textbf{open disk} or \textbf{circular neighborhood} of a point $z_0$ is the set of all points $z$ which satisfy the inequality
\[|z-z_0| < p,\]
where $p$ is a positive real number.  This set consists of every point that lies inside the circle of radius $p$ around the center $z_0$.
\end{defn}

\begin{example}
The solution sets of the inequalities
\[|z-2| < 5, \;\;\; |z+i| < \frac{1}{2}, \;\;\; |z| < 8\]
are open disks centered at $2$, $-i$, and $0$ respectively.
\end{example}

A frequently used neighborhood is the \textit{open unit disk}:

\begin{defn}
The \textbf{open unit disk}, denoted $\mathbb{D}$ is as follows:
\[\mathbb{D} := \{z : |z| < 1\}.\]
\end{defn}

We will now define several closely related topological terms regarding sets.  We start with two terms in a single definition --- a two-for-one special, if you will.
\begin{defn}
For any set $S$, a point $z_0$ is called an \textbf{interior point} of $S$ if there is some open disk centered at $z_0$ which is completely contained in $S$.  If every point in $S$ is an interior point of $S$, we describe $S$ as an \textbf{open set}. 
\end{defn}

\begin{defn}
An open set $S$ is said to be \textbf{connected} if every pair of points can be joined by a curve which does not leave the set.  Alternatively, $S$ is connected if for all $p,q \in S$ there exists a finite collection of line segments contained in $S$ which join $p$ and $q$
\end{defn}

Essentially, a set is connected if it is a "single piece", geometrically speaking.  A \textit{convex set} is slightly more than this:

\begin{defn}
A set $S \in \mathbb{C}$ is said to be \textbf{convex} if
\[tp+(1-t)q \in S\]
for all $p,q \in S$ and for all $0 \leq t \leq 1$.  More simply, a set $S$ is convex if, for every pair of points in $S$, the line connecting the two points is contained within $S$.
\end{defn}

Note that, intuitively, all convex sets are connected, but a connected set is not necessarily convex.  Figure 2 shows this with geometric representations of sets --- (a) is both convex and connected, (b) and (c) are connected but not convex, and (d) is neither connected nor convex.

\begin{figure}[h]
\centering
\includegraphics[scale=0.5]{images/convexvconnected.png}
\caption{Convex vs. Connected Sets}
\end{figure}

\begin{defn}
A point $z_0$ is said to be a \textbf{boundary point} of a set $S$ if every neighborhood of $z_0$ contains at least one point of $S$ and at least one point not in $S$.  Simply put, $z_0$ is a boundary point if it exists on the edge of the set.  The set of all boundary points is called the \textbf{boundary} of $S$, and we will denote this $\partial(S)$.
\end{defn}

\begin{defn}
A set $S$ is said to be \textbf{closed} if it contains all of its boundary points.  Equivalently, $S$ is closed if its complement $\mathbb{C} \ S$ is open.
\end{defn}

\begin{example}
Let $\overbar{\mathbb{D}} = \{z : |z| \leq 1\}$.  This is a closed set, as it contains its boundary $\partial(\mathbb{D}) = \{z : |z| = 1\}$.  $\overbar{\mathbb{D}}$ is known as the \textit{closure of $\mathbb{D}$} or the \textit{closed unit disk}.
\end{example}

The final topological concepts we need for $\mathbb{C}$ are those of \textit{boundedness} and \textit{compactness}.

\begin{defn}
A set of points $S$ is said to be \textbf{bounded} if there exists some $r \in \mathbb{R}^+$ such that $|z| < r$ for every $z$ in $S$.  Equivalently, $S$ is bounded if it is contained in any neighborhood of the origin.  $S$ is \textbf{unbounded} if it is not bounded.
\end{defn}

\begin{defn}
A set that is both closed and bounded is said to be \textbf{compact}.
\end{defn}

We have sufficiently described the complex space $\mathbb{C}$ as an extension of the real numbers.  Furthermore, we have outlined a number of topological concepts for $\mathbb{C}$.  Later on, in the section on Conformal Mappings, we explore concepts from complex analysis much deeper --- specifically, we will describe analytic functions of the complex variable from a geometric standpoint.
\newpage
\subsection{Eigenvalues}
\input{eigenvalues}
\newpage
\subsection{Linear Transformations}
\input{LTs}

\newpage
\section{Analytic Functions and Conformal Mappings}
\input{functions}

\newpage
\section{Banach and $l^p$ Spaces}
The main results from \cite{main} which we are exploring in this paper deal with a special type of space called a \textit{Banach space} --- recall we very briefly mentioned Banach spaces at the end of section 2.1.  More specifically, a special subset of Banach spaces, called $l^p$ spaces, are used extensively in \cite{main}.

\begin{defn}
A metric space $(X,d)$ is called \textbf{complete} if every Cauchy sequence in $X$ converges to some in $E$.
\end{defn}

\begin{defn}
A normed vector space $(V, ||\cdot||)$ is called a \textbf{Banach space} if and only if $V$ is complete in the metric induced from the norm.
\end{defn}

Our first example of a Banach space requires a result known as the Bolzano-Weierstrass theorem to prove completeness.

\begin{theorem}[Bolzano-Weierstrass]
Every bounded sequence has a convergent subsequence.
\end{theorem}

\begin{proof}
Suppose $(s_n)$ is a bounded sequence.  By Theorem 2.14, $(s_n)$ has a monotonic subsequence, $(s_{n_k})$.  Since $(s_n)$ is a bounded sequence, $(s_{n_k})$ is bounded as well.  Since all bounded monotonic subsequences converge, we have that $(s_{n_k})$ converges.
\end{proof}

We now have all the necessary tools in order to easily show certain spaces are complete.

\begin{example}
Consider the normed vector spaces $\mathbb{Q}, \mathbb{R}, \mathbb{C}$.
\begin{itemize}
\item $\mathbb{R}$ with the standard Euclidean norm is a Banach space \cite{nthubanach}. 

We already know that $\mathbb{R}$ is a normed vector space.  So to prove it is a Banach space, we must show it is complete.  That is, we must show that every Cauchy sequence in $\mathbb{R}$ converges to a point in $\mathbb{R}$.  This follows immediately from three previous theorems.  First, note every Cauchy sequence is bounded (Theorem 2.17).  By Bolzano-Weierstrass, every bounded sequence in $\mathbb{R}$ has a convergent subsequence (with limit in $\mathbb{R}$).  Finally, if a subsequence of a Cauchy sequence converges, then the whole sequence converges to the same point (Theorem 2.19).  Thus $\mathbb{R}$ is complete, and is hence a Banach space.

\item $\mathbb{C}$ with its standard norm is a Banach space.  Indeed, if $(z_n) = (x_n) + i(y_n)$ is Cauchy in $\mathbb{C}$, then $(x_n)$ and $(y_n)$ are Cauchy in $\mathbb{R}$.  Since $\mathbb{R}$ is complete, we have $(x_n)$ converges to $x$ and $(y_n)$ converges to $y$, $x, y \in \mathbb{R}$.  Thus we have that $(z_n)$ converges to $z = x + iy$.  Hence $\mathbb{C}$ is complete and is thus a Banach space.

\item The metric space $(\mathbb{Q}, d)$ with its usual metric is not complete.  This is because there are Cauchy sequences in $\mathbb{Q}$ which converge to irrational limits.  For example, 
\[(s_n) = \biggl ( 1 + \frac{1}{n} \biggr )^n\]
is contained in $\mathbb{Q}$, but converges to $e$ which is not in $\mathbb{Q}$.
\end{itemize}
\end{example}

The following lemma is often useful in proving that a metric space is complete.

\begin{lemma}
Let $(X,d)$ be a metric space.  If every Cauchy sequence has a convergent subsequence in $X$, then $(X,d)$ is complete.
\end{lemma}

Now we will provide a criterion for completion of a normed space in terms of series.

\begin{defn}
There are a number of terms we must define for series.
\begin{itemize}
\item If $(E, ||\cdot||)$ is a normed space then a \textbf{series} in $E$ is just the summation a sequence $(x_n)$ of terms $x_n \in E$ denoted \[\sum_{k=m}^{n}x_n;\]
\item The sum 
\[s_n = \sum_{k=m}^{n}x_n\]
is called the \textbf{nth partial sum} of the series;
\item The series \textbf{converges} if the sequence of partial sums has a limit.  That is, if there exists $s \in E$ such that
\[\lim_{n \to \infty} \biggl|\biggl|\biggl(\sum_{k=m}^{n}x_k\biggr) - s\biggr|\biggr| = 0.\]
If the sequence of partial sums has no limit in $E$, we say the series \textbf{diverges};
\item We say a series $\sum_{n=m}^\infty x_n$ is \textbf{absolutely convergent} if $\sum_{n=m}^\infty ||x_n|| < \infty$.  This is because $\sum_{n=m}^\infty ||x_n||$ is the summation of a series of real positive terms, so the sequence of partial sums either converges in $\mathbb{R}$ or increases to $\infty$.
\end{itemize}
\end{defn}

\begin{defn}
A series satisfies the \textbf{Cauchy criterion} if its sequence of partial sums forms a Cauchy sequence.
\end{defn}

\begin{theorem}[Series criterion for completion]
Let $(E, ||\cdot||)$ be a normed vector space.  $E$ is complete (and thus is a Banach space) if and only if each absolutely convergent series $\sum_{n=1}^\infty x_n$ of terms $x_n \in E$ is convergent in $E$.
\end{theorem}

\begin{proof}
We begin with the forward direction.  Suppose $E$ is complete and $\sum_{n=q1}^{n}||x_j|| < \infty$.  Then the individual partial sums of this series
\[S_n = \sum_{j=1}^{n}||x_j||\]
must satisfy Cauchy criterion.  That is, for any $\epsilon > 0$, there exists $N$ such that if $n, m \geq N$, then $|S_n - S_m| < \epsilon$.  So suppose $n > m \geq N$.  Then,
\[|S_n - S_m| = \biggl|\sum_{j=1}^{n} ||x_j|| - \sum_{j=1}^{m} ||x_j||\biggr| = \sum_{j = m+1}^{n} ||x_j|| < \epsilon.\]
Now, consider the partial sums $s_n = \sum_{j=1}^{n} x_j$.  If $n>m\geq N$, then
\[||s_n - s_m|| = \biggl|\biggl|\sum_{j=1}^{n} x_j - \sum_{j=1}^{m} x_j\biggr|\biggr| = \biggl|\biggl|\sum_{j = m+1}^{n} x_j\biggr|\biggr| \leq \sum_{j=m+1}^{n} ||x_j|| < \epsilon.\]
Hence the sequence $(s_n)$ is Cauchy in $E$.  Since $E$ is complete, $\lim_{n \to \infty} s_n$ exists in $E$, and so $\sum_{n=1}^{\infty} x_n$ converges.  This completes the forward direction.

For the other direction, assume all absolutely convergent series in $E$ are convergent.  Let $(u_n)$ be a Cauchy sequence in $E$.  Then we can find $n_1 > 0$ such that if $n,m \geq n_1$ then 
\[||u_n - u_m|| < \frac{1}{2}.\]
We can also find $n_2 > 1$ such that if $n,m > n_2$ then
\[||u_n - u_m|| < \frac{1}{4} = \frac{1}{2^2}.\]
Without loss of generality we can assume that $n_2 > n_1$.  By continuing to pick $n_j$ in this way, we can find $n_1 < n_2 < n_3 < \cdots$ such that if $n,m \geq n_j$ then
\[||u_n - u_m|| < \frac{1}{2^j}.\]
Now, consider the series $\sum_{j=1}^{\infty}x_j = \sum_{j=1}^{\infty}(u_{n_{j+1}} - u_{n_j})$ (Note this is a subseries of $(u_n)$.  Then,
\begin{align*}
\sum_{j=1}^{\infty}||x_j|| &= \sum_{j=1}^{\infty}||u_{n_{j+1}} - u_{n_j}||\\
&\leq \sum_{j=1}^{\infty} \frac{1}{2^j}\\
&= 1 < \infty.
\end{align*}
Thus $\sum_{j=1}^{\infty}x_j$ is absolutely convergent and, by our assumption, must also be convergent.  Thus its sequence of partial sums
\[s_J = \sum_{j=1}^{J}(u_{n_{j+1}} - u_{n_j}) = u_{n_{J+1}} - u_{n_1}\]
has a limit in $E$.  Thus,
\[\lim_{J \to \infty} u_{N_{J+1}} = u_{n_1} + \lim_{J \to \infty} (u_{N_{J+1}} - u_{n_1})\]
exists in $E$.  We have shown that $(u_n)$ has a convergent subsequence in $E$.  By Lemma 4.4, we finally have that $E$ is complete.
\end{proof}

We now look at a special family of Banach spaces known as $l^p$-spaces.  $l^p$-spaces are the main type of space we will be working in in the main results section of the paper.

\begin{defn}
For $1 \leq p < \infty$, $l^p$ denotes the set of all sequences $x = \{a_n\}_{n=1}^{\infty}$ which satisfy
\[\sum_{n=1}^{\infty} |a_n|^p < \infty.\]
If $p = \infty$, we define $p^{\infty}$ as the set of all sequences $x = \{a_n\}_{n=1}^{\infty}$ which satisfy
\[\sup_{n \geq 1} |a_n| < \infty\]
\end{defn}

In other words, for $1 \leq p < \infty$, $l^p$ is the set of all sequences whose series converge when each of its individual elements is raised to the $p^{\textrm{th}}$ power.  $l^{\infty}$ is the set of all sequences whose limit is not infinity --- that is, the set of all bounded sequences.  We commonly combine $l^p$ with the following norms, called the \textit{$p$-norms}:

For $x = (a_n) \in l^p$, define
\[||x||_p = \biggl(\sum_{n=1}^{\infty}|a_n|^p\biggr)^{1/p}, \;\; \textrm{ for } p < \infty;\]
\[||x||_{\infty} = \sup_{n \geq 1} |a_n|.\]

It is very easy to see that the $l^p$ spaces are nested; as $p$ increases, $l^p$ become more permissive.  That is, 
\[l^1 \subset l^2 \subset l^3 \subset \cdots \subset l^{\infty}.\]

\begin{example}
We demonstrate the nesting of $l^p$.
Consider the harmonic sequence $(s_n) = \frac{1}{n}$.  It is well known that
\[\sum_{n=1}^{\infty} \biggl|\frac{1}{n}\biggr| = \infty.\]
Thus $(s_n) \notin l^1$.\\
However, 
\[\sum_{n=1}^{\infty} \biggl|\frac{1}{n}\biggr|^2 = \sum_{n=1}^{\infty} \biggl|\frac{1}{n^2}\biggr| < \infty.\]
Thus $(s_n) \in l^2$.  Similarly, $(s_n) \in l^p$ for $p > 2$.
\end{example}

We can also easily see that the $p$-norms are nested as well.  We show this by drawing the unit ball
\[\{x \in \mathbb{R}^2 : ||x||_p \leq 1\}\]
for each $p$-norm in $\mathbb{R}^2$ (Figure 3).

\begin{figure}[h]
\centering
\includegraphics[scale=0.5]{images/p-norms.png}
\caption{The unit ball of various $p$-norms \cite{pnormpic}}
\end{figure}

We now show that $l^p$ is a Banach space under its appropriate p-norm.  We do this in several steps, first showing $l^p$ is a vector space.  Then, we prove the properties of a norm hold for the p-norms.  Finally, we show that $l^p$ is complete.  Note we often omit the cases for $p=1$ and $p=\infty$, as those cases will sometimes require separate arguments which are simple and direct.

\begin{prop}
$l^p (1 \leq p \leq \infty)$ is a vector space.
\end{prop}

\begin{proof}
If $x \in l^p$ and $c \in \mathbb{C}$, then clearly $cx \in l^p$.  Then, if $x,y \in l^p$, since $|x_n + y_n| \leq (2|x_n|^p) + (2|y_n|^p)$, we also have that $x+y \in l^p$.  So addition and scalar multiplication can be defined on $l^p$.  Now, algebraic operations on $l^p$ are defined componentwise; for example, $x+y$ is the sequence whose $n$th element is $x_n + y_n$.  Since all operations are defined this way, the algebraic rules for $l^p$ are simply inherited from $\mathbb{C}$.
\end{proof}

\begin{prop}
$||\cdot||_p$ is a norm on $l^p$
\end{prop}

The first two properties of a norm are obvious.  To prove the triangle inequality, though, we must first prove an inequality known as H{\"o}lder's inequality.  For H{\"o}lder's inequality, we need the following Lemma, which we will take as given.

\begin{lemma}
Suppose $1 < p < \infty$ and q is defined by $\frac{1}{p} + \frac{1}{q} = 1$.  Then,
\[ab \leq \frac{a^p}{p} + \frac{b^q}{q} \qquad \textrm{ for } a,b \geq 0.\]
\end{lemma}

\begin{example}[of previous Lemma]
Suppose $p = 3$.  Then $q = \frac{3}{2}$ since $\frac{1}{p} + \frac{1}{q} = 1$.  Consider several values for $a$ and $b$.
\begin{itemize}
\item $a = 1, b=1$\\
$(1)(1) = 1.$  $\frac{1^3}{3} + \frac{1^{3/2}}{3/2} = \frac{1}{3} + \frac{2}{3} = 1$ 
\item $a = 4, b=9$\\
$(4)(9) = 36$.  $\frac{4^3}{3} + \frac{9^{3/2}}{3/2} = \frac{64}{3} + \frac{54}{3} = \frac{118}{3} > 36$.
\item $a = 20, b=64$\\
$(20)(64) = 1280$.  $\frac{20^3}{3} + \frac{64^{3/2}}{3/2} = \frac{8000}{3} + \frac{1024}{3} = \frac{9024}{3} = 3008$.
\end{itemize}
\end{example}

\begin{lemma}[H{\"o}lder's Inequality]
Let $x \in l^p, y \in l^q,$ where $1 \leq p < \infty$ and
$\frac{1}{p} + \frac{1}{q} = 1.$
Then,
\[||xy||_1 = \sum_{n=1}^{\infty} |a_nb_n| \leq ||(a_n)_n||_p||(b_n)_n||_q.\]
\end{lemma}

This inequality also shows that $xy \in l^1$ under the same conditions.  Note that if $p=1$, we interpret this to mean that $q = \infty$.

\begin{proof}
If $p=1$ and $q= \infty$, this is obvious.  So, suppose $p > 1$.  Let $A$ and $B$ be as follows:
\begin{align*}
A \; &= \; ||x||_p = \biggl(\sum_{n=1}^{\infty} |a_n|^p\biggr)^{1/p}\\
B \; &= \; ||y||_q = \biggl(\sum_{n=1}^{\infty} |b_n|^q\biggr)^{1/q}
\end{align*}
If either $A=0$ or $B=0$, the inequality is obviously true; both sides must equal 0.  So, assume $A \neq 0, B\neq 0$ and use Lemma 4.13 with $a = |a_n|/A$ and $b = |b_n|/B$ to see
\begin{align*}
\frac{|a_nb_n|}{AB} &\leq \frac{1}{p}\frac{|a_n|^p}{A^p} + \frac{1}{q}\frac{|b_n|^1}{B^q}\\
\textrm{So } \sum_{n=1}^{\infty} \frac{|a_nb_n}{AB} &\leq \frac{1}{p}\sum_{n=1}^{\infty}\frac{|a_n|^p}{A^p} + \frac{1}{q}\sum_{n=1}^{\infty}\frac{|b_n|^1}{B^q}\\
&= \frac{1}{p} + \frac{1}{q}\\
&= 1.
\end{align*}
Hence
\[\sum_{n=1}^{\infty}|a_nb_n| \leq AB = ||x||_p||y||_q.\]
\end{proof}

\begin{lemma}[Minkowski's inequality or triangle inequality for $l^p$]
If $x,y \in l^p (1 \leq p \leq \infty)$, then $x+y \in l^p$ and
\[||x+y||_p \leq ||x||_p + ||y||_p\]
\end{lemma}

\begin{proof}
This inequality is trivial for $p=1$ and $p=\infty$, so suppose $1 < p < \infty$.  We already know $(x_n + y_n)_n \in l^p$ (Proposition 4.11).  For the inequality, observe that
\begin{align*}
\sum_n |x_n + y_n|^p &= \sum_n |x_n + y_n|\,|x_n + y_n|^{p-1}\\
&= \sum_n |x_n|\,|x_n + y_n|^{p-1} + \sum_n |y_n|\,|x_n + y_n|^{p-1}.
\end{align*}
Let $(a_n) = |x_n|$ and $(b_n) = |x_n + y_n|^{p-1}$.  So $(a_n) \in l^p$.  Also, $(b_n) \in l^p$ since
\begin{align*}
\sum_n (b_n)^q &= \sum_n |x_n + y_n|^{(p-1)q}\\
&= \sum_n |x_n + y_n|^p \qquad\textrm{ since } \frac{1}{p} + \frac{1}{q} = 1\\
&< \infty \qquad\textrm{ since we know } (x_n + y_n)_n \in l^p.\\
\end{align*}
Now, from H{\"o}lder's inequality, we have
\begin{align*}
\sum_n |x_n|\,|x_n+y_n|^{p-1} &\leq \biggl(\sum_n |x_n|^p\biggr)^{1/p} \biggl(\sum_n |x_n + y_n|^{(p-1)q}\biggr)^{1/q}\\
&= (||x||_p\,||x + y||_p)^{p/q}
\end{align*}
Similarly, we have
\[\sum_n|x_n|\,|x_n + y_n|^{p-1} \leq ||y||_p\bigl(||x+y||_p\bigr)^{p/q}.\]

If we add the two inequalities, we are left with
\[\bigl(||x+y||_p\bigr)^p \leq ||x||_p\bigl(||x + y||_p\bigr)^{p/q} + ||y||_p\bigl(||x+y||_p\bigr)^{p/q}.\]

Now, if $||x+y||_p = 0$, then Minkowski's inequality is true since a property of norms is that $||\cdot|| \geq 0$.  So suppose $||x+y||_p \neq 0$.  Then we can divide both sides of the inequality by $\bigl(||x+y||_p\bigr)^{p/q}$ to get
\[\bigl(||x+y||_p\bigr)^{p-p/q} \leq ||x||_p + ||y||_p.\]
Since, $p-\frac{p}{q} = 1$, we have achieved our goal.
\end{proof}

At this point, we have that $l^p$ is a normed vector space for $1 \leq p \leq \infty$.  To show that $l^p$ is a Banach space, all that remains is to show that $l^p$ is complete.  The proof of this requires some awkward notation, since the elements of $l^p$ are sequences themselves --- so, a sequence of elements in $l^p$ is a sequence of sequences.  So in a sequence of sequences in $l^p$, we will use a superscript to denote the $n$th sequence --- i.e. $x^{(n)}$ is the $n$th sequence.

\begin{theorem}
$l^p$ is a complete under the norm
\[||x||_p = \biggl(\sum_{n=1}^{\infty}|x_n|^p\biggr)^{1/p}\]
for $x \in l^p, 1 < p < \infty$.
\end{theorem}

\begin{proof}
Suppose $x^{(n)} \in l^p$ be a Cauchy sequence.  Let $\epsilon > 0$.  Since $x^{(n)}$ is Cauchy, there exists $n_0 \in \mathbb{N}$ such that if $n,m > n_0$, then
\[\sum_{j=1}^{\infty} |x_j^{(n)} - x_j^{(m)}|^p < \epsilon ^p.\]
So, for each fixed $j \in \mathbb{N}$,
\[|x_j^{(n)} - x_j^{(m)}| \leq \biggl(\sum_{j=1}^{\infty} |x_j^{(n)} - x_j^{(m)}|^p\biggr)^{1/p} < \epsilon .\]
That is, $(x_j^{(n)})$ is a Cauchy sequence in $\mathbb{C}$.  Since $\mathbb{C}$ is complete, these sequences have limits in $\mathbb{C}$.  Define
\[x_j = \lim_{n \to \infty} x_j^{(n)}.\]
Let $x = (x_j)$.  So, we have $x^{(n)} \to x$.  We must verify that $x \in l^p$.  Observe that if we arbitrarily fix $N \in \mathbb{N}$,
\begin{align*}
\sum_{j=1}^{N} |x_j|^p &= \lim_{n \to \infty} \sum_{j=1}^{N} \bigl|x_j^{(n)}\bigr|^p\\
&\leq \limsup_{n \to \infty} ||x^{(n)}||^p.
\end{align*}
Since $x^{(n)} \in l^p$, this is sufficient to show that $x = (x_j) \in l^p$.
\end{proof}

\newpage
\section{Chaos}
The following is a well known criterion for chaos, known as the Eigenvalue Criterion.  [2,3] provide proofs for the Criterion, and [9,10,12,14,15] provide examples using the Criterion.

\begin{theorem}
Let $T:X \rightarrow X$ be an operator on a separable complex Banach space $X$.  Consider the subspaces
\[X_0 := \rm{Span}\{x \in X : T(x) = \lambda X \textrm{ for some }\lambda \in \mathbb{C} with |\lambda| < 1\}, \]
\[Y_0 := \rm{Span}\{x \in X : T(x) = \lambda X \textrm{ for some } \lambda \in \mathbb{C} with |\lambda| > 1\},\]
\[Z_0 := \rm{Span}\{x \in X : T(x) = e^{\alpha \pi i}x \textrm{ for some } \alpha \in \mathbb{Q}\}.\]
If $X_0, Y_0,$ and $Z_0$ are all dense in $X$, then $T$ is chaotic.
\end{theorem}

Since the set of eigenvalues $\sigma_p(B) = \mathbb{D}$ in our framework, this Criterion says that $\varphi(B)$ is chaotic on $l^p$ if and only if $\varphi(\mathbb{D})$ intersects the unit circle.

\newpage
\section{Main results}
\begin{lemma}
Let $\varphi$ be a linear fractional transformation as in (2) and $|d| > |c|$.  Then $\varphi(\mathbb{D})$ is the disc $P+r\mathbb{D}$ with center P and radius r given by
\[P = \frac{b\conj{d} - a\conj{c}}{|d|^2 - |c|^2}, \;r = \frac{bc - ad}{|d|^2 - |c|^2}.\]
\end{lemma}

asdf
In [10], DeLaubenfels and Emamirad showed that, for a non-constant polynomial $P(z)$, $P(B)$ (where $B$ is the backwards shift operator) is chaotic on $l^p, 1\leq p \leq \infty$ whenever $P(\mathbb{D})$ intersects the unit disc.  We provide a generalization of this result which can be applied to Linear Fractional Transformations.

\begin{theorem}
Let $\varphi$ be a LFT with $c \neq 0$ and $|d| > |c|$.  The operator $\varphi(B)$ is chaotic if and only if 
\[\bigl\lvert |d|^2 - |c|^2 - |b\conj{d} - a\conj{c}| < |bc - ad| \bigr\rvert.\]
\end{theorem}

\begin{proof}
We showed in Lemma 2 that $\varphi(\mathbb{D}) = P+r\mathbb{D}$ with center $P$ and radius $r$ given where
\[P = \frac{b\conj{d} - a\conj{c}}{|d|^2 - |c|^2}, \;r = \frac{|bc - ad|}{|d|^2 - |c|^2}.\]
Theorem 1, the Eigenvalue Criterion, showed that $\varphi(B)$ is chaotic on $l^p$ if and only if $\varphi(\mathbb{D})$ intersects the unit circle.  So, we have that $\varphi(B)$ is chaotic if and only if the disc $P+r\mathbb{D}$ intersects the unit circle.

In order for the disc to intersect the unit disc, we have two possibilities: the center of the disc $P$ is contained within the unit circle, or $P$ is outside the closed unit disc.  If $P$ is inside the unit disc, then $|P| + |r| > 1$; if $P$ is outside the closed unit disc, then we must have $|P| - |r| < 1$.

These conditions lead to 
\[-|r| < 1 - |P| < |r|.\]
After substituting in the values of $P$ and $r$, we have
\[-\frac{|bc - ad|}{|d|^2 - |c|^2} < 1 - \biggl\lvert\frac{|b\conj{d} - a\conj{c}|}{|d|^2 - |c|^2}\biggr\rvert < \frac{|bc - ad|}{|d|^2 - |c|^2}.\]
Multiplying by $|d|^2 - |c|^2$ gives
\[-|bc-ad| < |d|^2 - |c|^2 - |b\conj{d} - a\conj{c} < |bc - ad|.\]
So finally, we have
\[\bigl\lvert|d|^2 - |c|^2 - |b\conj{d} - a\conj{c}|\bigr\rvert < |bc-ad|.\]

So, we have shown that if $\varphi(B)$ is chaotic, then $\bigl\lvert|d|^2 - |c|^2 - |b\conj{d} - a\conj{c}|\bigr\rvert < |bc-ad|.$  The other direction is completely analogous.  It requires the exact same algebra, done in reverse, to show that if $\bigl\lvert|d|^2 - |c|^2 - |b\conj{d} - a\conj{c}|\bigr\rvert < |bc-ad|$, then $P+r\mathbb{D}$ intersects the unit circle, and thus $\varphi(B)$ is chaotic.
\end{proof}

\newpage  
\begin{thebibliography}{99}
\singlespacing
\bibitem{Edgar} Edgar, Gerald, \emph{Measure, Topology, and Fractal Geometry}, Second Ed., Springer Science+Business Media, 2008.

\bibitem{Saff} Saff, E.B. and Snider, A.D., \emph{Fundamentals of Comples Analysis with Applications to Engineering and Science}, Third Ed., Pearson Education, 2003.

\bibitem{Kaplansky} Kaplansky, Irving, \emph{Set Theory and Metric Spaces}, reprint, American Mathematical Society, 2001.

\bibitem{modulusimage} Alexandrov, Oleg, \emph{Illustration of Complex Conjugate}, 2007.

\bibitem{Johnson} Johnson, Lee W., \emph{Introduction to Linear Algebra}, Fifth Ed., Pearson Education, 2001.

\bibitem{main} Jim{\"e}nez-Mungu{\"i}a, Ronald R., Gal{\"a}n, V{\"i}ctor J., Mart{\"i}nez-Gim{\"e}nez, and F{\"e}lix, Peris, Alfredo, \emph{Chaos for Linear Fractional Transformations of Shifts}, Elsevier B.V., 2016

\bibitem{Abbott} Abbott, Stephen, \emph{Understanding Analysis}, Second Ed., Springer, 2015

\bibitem{nthubanach} Chen, Kuo-Chang, \emph{Introduction to Banach Spaces}, \url{http://www.math.nthu.edu.tw/~kchen/teaching/5131week1.pdf}

\end{thebibliography}

\addcontentsline{toc}{section}{References}

\end{document}
